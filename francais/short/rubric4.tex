%%% rubric.tex --- Example of using CurVe.

%% Copyright (C) 2000, 2001, 2002, 2003, 2004, 2005, 2010 Didier Verna.

%% Author:        Didier Verna <didier@lrde.epita.fr>
%% Maintainer:    Didier Verna <didier@lrde.epita.fr>
%% Created:       Thu Dec 10 16:04:01 2000
%% Last Revision: Mon Dec  6 11:04:22 2010

%% This file is part of CurVe.

%% CurVe may be distributed and/or modified under the
%% conditions of the LaTeX Project Public License, either version 1.1
%% of this license or (at your option) any later version.
%% The latest version of this license is in
%% http://www.latex-project.org/lppl.txt
%% and version 1.1 or later is part of all distributions of LaTeX
%% version 1999/06/01 or later.

%% CurVe consists of the files listed in the file `README'.


%%% Commentary:

%% Contents management by FCM version 0.1.


%%% Code:


\begin{rubric}{Compétences}
                \subrubric{Langues pratiquées}
                    \entry*
                    
                Français, Espagnol: parlé couramment. Anglais : lu,
                    écrit, parlé.
                \subrubric{Compétences informatiques}
                    \entry*
                    
                Maîtrise des méthodes et outils d'édition électronique
                    selon le standard XML TEI/P5. Utilisation courante des
                    langages \LaTeX, xml, xsl, css et html. Utilisation courante
                    du langage python et des expressions régulières.
                    Compréhension et utilisation du javascript et du XQuery.
                    Expérience dans la construction de bases de données XML.
                    Expérience avec le framework de développement web
                    flask.
                    \entry*
                    
                Utilisation courante des logiciels permettant la
                    création de modèles d'apprentissage supervisé (Ocropy,
                    reconnaissance de caractères; Pie, analyse
                    grammaticale).
                    \entry*
                    
                Maîtrise du logiciel de contrôle de version
                    git.\end{rubric}


%%% rubric.tex ends here

%%% Local Variables:
%%% mode: latex
%%% TeX-master: "raw"
%%% End: