%%% rubric.tex --- Example of using CurVe.

%% Copyright (C) 2000, 2001, 2002, 2003, 2004, 2005, 2010 Didier Verna.

%% Author:        Didier Verna <didier@lrde.epita.fr>
%% Maintainer:    Didier Verna <didier@lrde.epita.fr>
%% Created:       Thu Dec 10 16:04:01 2000
%% Last Revision: Mon Dec  6 11:04:22 2010

%% This file is part of CurVe.

%% CurVe may be distributed and/or modified under the
%% conditions of the LaTeX Project Public License, either version 1.1
%% of this license or (at your option) any later version.
%% The latest version of this license is in
%% http://www.latex-project.org/lppl.txt
%% and version 1.1 or later is part of all distributions of LaTeX
%% version 1999/06/01 or later.

%% CurVe consists of the files listed in the file `README'.


%%% Commentary:

%% Contents management by FCM version 0.1.


%%% Code:

\begin{rubric}{Enseignement et formations dispensées en HN}
                                \entry*[]
                            Cours de méthodologie de la recherche et
                        introduction à \LaTeX. Intitulé du cours: \enquote{La
                        publication scientifique: enjeux économiques,
                        juridiques, éditoriaux et techniques} (2018-2020).
                    
                                \hspace{-1cm}
                                \entry*
                            Cours d'humanités numériques (10h): introduction à
                        l'analyse grammaticale du castillan médiéval (2020).
                    
                                \hspace{-1cm}
                                \entry*
                            Animation de la session 2020 du MOOC \textit{Digital
                        Scholarly Editions: Manuscripts, Texts and TEI
                        Encoding} organisé par Marjorie Burghart.
                    
                                \hspace{-1cm}
                                \entry*
                            Co-organisation et co-animation de la formation \href{https://github.com/gabays/Cours\_COSME\_2019}{XSLT
                        COSME 2019}. 
                    \end{rubric}

%%% rubric.tex ends here

%%% Local Variables:
%%% mode: latex
%%% TeX-master: "raw"
%%% End: