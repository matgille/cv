%%% rubric.tex --- Example of using CurVe.

%% Copyright (C) 2000, 2001, 2002, 2003, 2004, 2005, 2010 Didier Verna.

%% Author:        Didier Verna <didier@lrde.epita.fr>
%% Maintainer:    Didier Verna <didier@lrde.epita.fr>
%% Created:       Thu Dec 10 16:04:01 2000
%% Last Revision: Mon Dec  6 11:04:22 2010

%% This file is part of CurVe.

%% CurVe may be distributed and/or modified under the
%% conditions of the LaTeX Project Public License, either version 1.1
%% of this license or (at your option) any later version.
%% The latest version of this license is in
%% http://www.latex-project.org/lppl.txt
%% and version 1.1 or later is part of all distributions of LaTeX
%% version 1999/06/01 or later.

%% CurVe consists of the files listed in the file `README'.


%%% Commentary:

%% Contents management by FCM version 0.1.


%%% Code:


\begin{rubric}{}
                    \subrubric{Publications}
                   
                
                    
                    \entry*
                
                        \enquote{Educación del príncipe, nobleza y caballería en el \textit{De regimine principum} castellano (segundo y tercer
                            libro)}, \textit{Librosdelacorte} 22, 2021, pp.
                        285-308. DOI:\href{https://doi.org/10.15366/ldc2021.13.22.010}{10.15366/ldc2021.13.22}
                    
                    
                    \entry*
                
                        \enquote{L'évolution du Regimiento de los prínçipes (1345-1494),
                            conditionnée par le pouvoir politique ?} in \textit{Écritures du Pouvoir}, 2, éd. Véronique Lamazou-Duplan, Ausonius
                        éditions, 2019, Scripta Medievalia, pp. 137-148. \href{https://hal.archives-ouvertes.fr/hal-02369116}{HAL.}
                    
                \vspace{1cm}
                \subrubric{Articles de conférences}
            
                    
                    \entry*
                Simon Gabay, Thibault Clérice, Jean-Baptiste Camps, Jean-Baptiste Tanguy,
                        Matthias Gille-Levenson. \enquote{Standardizing linguistic data: method and
                                tools for annotating (pre-orthographic) French.}, Proceedings
                        of the 2nd International Digital Tools and Uses Congress (DTUC '20), Oct 2020,
                        Hammamet, Tunisia, hal-03018381. DOI:\href{https://dx.doi.org/10.1145/3423603.3423996}{10.1145/3423603.3423996}
                \vspace{1cm}
                \subrubric{Communications dans des congrès internationaux}
            
                    
                    \entry*
                \textit{\textit{TEICollator: a TEI to TEI workflow}} (\textit{long
                            paper}), TEI2022: Text as data, Newcastle (Angleterre) 12-16 septembre
                        2022.
                    
                    \entry*
                \textit{TEICollator: une chaîne de traitement ecdotique
                            semi-automatisée}, 30e Congrès International de Linguistique et de
                        Philologie Romane, Université de La Lagune, Canaries (Espagne), 4-9 juillet 2022.
                            \href{https://hal.archives-ouvertes.fr/hal-03715059}{Dépôt
                        HAL}.
                    
                    \entry*
                Matthias Gille Levenson, Gwenaëlle Patat (ordre alph.). \textit{Programming
                            Historian en français : Faire communauté pour le partage de ressources
                            éducatives libres sur les méthodes numériques en sciences humaines et
                            sociales francophones}, Colloque Humanistica 2022, Mai 2022, Montréal,
                        Canada. \href{https://hal.archives-ouvertes.fr/hal-03672420}{hal-03672420}
                    
                    \entry*
                Matthias Gille Levenson, Olivier Brisville-Fertin, María Díez Yáñez, Simon
                        Gabay, \enquote{Construcción de un corpus de evaluación de la anotación
                                léxico-gramatical del castellano medieval (siglos XIII-XV)},
                        HDH2021: scire vias, Saint Jacques de Compostelle, Espagne, oct. 2021. \href{https://gitlab.huma-num.fr/mgillelevenson/paper_hdh_lematizacion}{Lien
                            vers le dépôt git}.
                    
                    \entry*
                Simon Gabay, Lucie Rondeau Du Noyer, Matthias Gille Levenson, Ljudmila
                        Petkovic et Alexandre Bartz, \textit{\textit{Quantifying the Unknown: How many
                                manuscripts of the marquise de Sévigné still exist?}}, DH2020,
                        Ottawa, Canada. \href{https://hal.archives-ouvertes.fr/hal-02898929/document}{\textit{Abstract}} et \href{http://dx.doi.org/10.17613/2pwa-0f46}{diapositives}.
                    
                    \entry*
                Paul Bertrand, Ariane Pinche, Matthias Gille Levenson et Margot Ferrand,
                            \textit{COSME² - Complexities: 30 Years Of Research Of
                                Medievalists DH Concerning A Thousand Years Of Medieval
                            Sources} (poster). Congrès \enquote{\textit{Digital Humanities
                            2019}} (DH2019), du 9 au 12 juillet 2019, Utrecht, Pays-Bas. \href{https://zenodo.org/record/3275597}{\includegraphics[scale=0.55]{img/zenodo3275597.png}}
                    
                \vspace{1cm}
                \subrubric{Communications à des colloques ou des journées d'étude}
            
                    
                    \entry*
                
                        \textit{La transcription automatisée, un changement de paradigme
                            ecdotique?}, colloque \textit{Trilobe II}, 29-30 septembre 2022,
                        ENS Lyon.
                    
                    \entry*
                Matthias Gille Levenson, Sofia Papastamkou et Célian Ringwald (ordre
                        alph.). \textit{Programming Historian : un lieu de collaborations et
                            d’interactions multiples}, Colloque DHNord 2022, juin 2022, en
                        ligne.
                    
                    \entry*
                \textit{Réception du roman, fiction et valeur du vrai dans trois traités de
                            littérature politique des \textsc{xiv} et
                                \textsc{xv}\textsuperscript{è} siècles en castillan}, Journée
                        d'études \enquote{Vérité et mensonge}, ENS de Lyon, 6 mai 2021.
                    
                    \entry*
                \textit{Describir y prescribir el liderazgo en la Castilla
                            bajomedieval}, Séminaire du Madrid Institute of Advanced Studies
                        (MIAS), co-animé avec Edward Holt, 22 mars 2021.
                    
                    \entry*
                
                        \textit{``E mayormente los reyes''. Educación del hijo y educación del príncipe en
                            el segundo libro del \textit{De regimine principum} castellano},
                            \textsc{I Seminario de Corte y Literatura}, Instituto Universitario La Corte
                        en España, UAM, 26-27 novembre 2021.
                    
                    \entry*
                
                        \textit{\textit{Editing a 15th century political treatise using the computer: a back
                                and forth between meaning and information}}, \textit{Iberian
                            Connections seminar}, Yale University, 12 novembre 2019. \href{https://iberian-connections.yale.edu/articles/editing-a-xvth-century-political-treatise-using-the-computer/}{Lien}.
                    
                    \entry*
                
                        Du concept au récit: théorie politique et
                                enxiemplo dans le \textit{Regimiento de los prínçipes} (III,
                            3). Colloque \enquote{Theorica 7: Politia}, ÉNS de Lyon, Lyon, 6
                        juin 2019. 
                    
                    \entry*
                \textit{La variance et ses enjeux dans l'édition numérique du \textit{Regimiento
                                de los príncipes}}. Journée d'étude Humanités Numériques,
                        Agorantic-Ciham, Avignon, 27 mai 2019.
                    
                    \entry*
                \textit{Construire l'édition numérique d'une tradition textuelle complexe:
                            propositions pour le \textit{Regimiento de los prínçipes}}. Atelier
                        IRPALL \enquote{La fabrique du texte}, Toulouse, 22 février
                        2019.
                    
                    \entry*
                \textit{L'évolution du \textit{Regimiento de los prínçipes} (1345-1494), au
                            service du pouvoir ?} Séminaire international de formation avancée:
                            \href{https://recherche.univ-pau.fr/fr/accueil/cpim.html}{Les cultures
                            politiques dans la Péninsule ibérique et au Maghreb},
                            Bielle, octobre 2018. 
            \vspace{1cm}
            \subrubric{Logiciels développés}
            
                    
                    \entry*
                \textbf{TeiCollator}: \href{https://github.com/matgille/tei-collator}{TeiCollator}
                        est un outil de collation semi-automatisée TEI to TEI qui s'appuie sur \href{https://pypi.org/project/collatex/}{CollateX} pour l'alignement
                        des témoins.
                    
                    \entry*
                \textbf{ClaViTranscr}: \href{https://github.com/matgille/transcription_virtual_keyboard}{ClaViTranscr} est un clavier virtuel qui permet d'aider au travail
                        d'établissement de vérités terrains dans le cadre de l'utilisation de logiciels
                        de transcription automatisée comme \href{https://gitlab.inria.fr/scripta/escriptorium}{eScriptorium}.
\end{rubric}

%%% rubric.tex ends here

%%% Local Variables:
%%% mode: latex
%%% TeX-master: "raw"
%%% End: