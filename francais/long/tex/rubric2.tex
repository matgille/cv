%%% rubric.tex --- Example of using CurVe.

%% Copyright (C) 2000, 2001, 2002, 2003, 2004, 2005, 2010 Didier Verna.

%% Author:        Didier Verna <didier@lrde.epita.fr>
%% Maintainer:    Didier Verna <didier@lrde.epita.fr>
%% Created:       Thu Dec 10 16:04:01 2000
%% Last Revision: Mon Dec  6 11:04:22 2010

%% This file is part of CurVe.

%% CurVe may be distributed and/or modified under the
%% conditions of the LaTeX Project Public License, either version 1.1
%% of this license or (at your option) any later version.
%% The latest version of this license is in
%% http://www.latex-project.org/lppl.txt
%% and version 1.1 or later is part of all distributions of LaTeX
%% version 1999/06/01 or later.

%% CurVe consists of the files listed in the file `README'.


%%% Commentary:

%% Contents management by FCM version 0.1.


%%% Code:

\begin{rubric}{Enseignement et formations dispensées}
                                \entry*[2022-2023]
                                
                            Janvier: participation à l'animation de la formation EnExDi 2023; séances
                        sur le HTR et la transcriptions de sources au format XML-TEI. Une
                        semaine.
                    
                                \entry*
                            Octobre: participation au séminaire du master HN de l'ENSSIB/Lyon 3/ENS de
                        Marianne Reboul; une séance en collaboration avec Ariane Pinche sur la
                        transcription automatisée. 3 heures.
                    
                                \entry*
                            Cours d'intoduction aux humanités numériques (L3-M1): De l'image au texte
                        annoté. Le cours a consisté en une introduction aux différentes méthodes de
                        traitement et d'étude des sources textuelles anciennes, en utilisant une même
                        source littéraire comme fil conducteur, la \textit{Breve relación de la
                            destruyción de las Indias} (1552) de Bartolomé de las Casas: courte
                        introduction à la bibliographie matérielle, OCR et travail sur l'apprentissage
                        supervisé, segmentation, annotation lexico-grammaticale. Un semestre.
                    
                                \entry*[2021-2022]
                                
                            Mai: animation d'un atelier doctoral à l'Universidad Complutense de Madrid
                        (école \enquote{\textit{Herramientas digitales}}) sur le HTR et eScriptorium
                        (quatre heures).
                    
                                \entry*
                            Animateur principal de la session 2022 du MOOC \textit{Digital Scholarly
                            Editions: Manuscripts, Texts and TEI Encoding} organisé par Marjorie
                        Burghart dans le cadre du Master Mondes Médiévaux de l'Université Lyon II.
                        Mars-avril 2022 (six séances).
                    
                                \entry*
                            Co-organisation et co-animation de la formation à la transcription
                        automatisée de sources manuscrites avec le logiciel \textit{eScriptorium},
                        avec María Díez Yáñez et Irene Salvo García, organisée à la Casa de Velázquez en
                        septembre 2021 (vingt participant.es présent.es, trente en distanciel), avec la
                        collaboration et la participation de Peter Stokes, Benajmin Kiessling, Leonor
                        Zozaya-Montes et de Belén Almeida (deux jours). \href{https://www.casadevelazquez.org/news/formation-seminaire-sur-la-transcription-automatisee-de-sources-manuscrites-avec-escriptorium/}{Lien vers la page de l'événement}.
                    
                                \entry*[2020-2021]
                                
                            Coorganisation et animation de la formation (en ligne) ODD, avec Emmanuelle
                        Morlock, Sarah Orsini et Ariane Pinche (trois demi-journées). \href{https://tei-odd-2021.sciencesconf.org/}{Lien vers la page de
                            l'événement}
                    
                                \entry*
                            Atelier \textit{Lemmatiser des textes et corriger l'annotation grâce a
                            l'apprentissage profond avec Pyrrha} coanimé avec Thibault Clérice,
                        Lucence Ing, Ariane Pinche, Simon Gabay et Jean-Baptiste Camps lors du congrès
                        Humanistica 2021 (quatre heures). \href{https://humanistica2021.sciencesconf.org/341021}{Lien vers la page de
                            l'atelier} et \href{https://hal.archives-ouvertes.fr/hal-03224112/file/abstract.pdf}{Abstract}.
                    
                                \entry*
                            Participation à l'animation de la session 2021 du MOOC \textit{Digital
                            Scholarly Editions: Manuscripts, Texts and TEI Encoding} organisé par
                        Marjorie Burghart. Mars-avril 2021 (six séances).
                    
                                \entry*
                            Animation d'une séance d'introduction à l'analyse grammaticale et à
                        l'utilisation de l'outil Pyrrha, dans le cadre du master HN de
                        l'ENSSIB.
                    
                                \entry*[2019-2020]
                                
                            Participation à l'animation de la session 2020 francophone du MOOC
                            \textit{Digital Scholarly Editions: Manuscripts, Texts and TEI Encoding}
                        organisé par Marjorie Burghart. Mars-avril 2020 (six séances)
                    
                                \entry*
                            Cours d'introduction à l'histoire culturelle de la Castille sous les
                        Trastamare. Thème: le pouvoir et sa légitimation (demi-semestre).
                    
                                \entry*
                            Cours d'introduction aux humanités numériques: l'analyse grammaticale du
                        castillan médiéval (demi-semestre).
                    
                                \entry*
                            Cours de méthodologie de la recherche. Intitulé du cours: \enquote{La
                            publication scientifique: enjeux économiques, juridiques, éditoriaux et
                            techniques}. Le cours a alterné entre présentation de la publication
                        scientifique sous ses différents prismes et initiation à la production de
                        documents scientifiques avec le logiciel/langage de composition \LaTeX (semestre
                        plein).
                    
                                \entry*[2018-2019]
                                
                            Organisation et animation de la formation \href{https://cosme.hypotheses.org/1117}{XSLT COSME 2019} en
                        collaboration avec trois autres doctorant.es/docteurs: Simon Gabay, Université de
                        Lausanne, Ariane Pinche, École nationale des Chartes, Jean-Paul Rehr, Université
                        Lyon II (deux jours). \href{https://github.com/gabays/Cours\_COSME\_2019}{Lien vers le dépôt de la formation}.
                    
                                \entry*
                            Cours de version classique, niveau Master (un semestre).
                    
                                \entry*
                            Cours de méthodologie de la recherche. Intitulé du cours: \enquote{La
                            publication scientifique: enjeux économiques, juridiques, éditoriaux et
                            techniques}. Le cours a alterné entre présentation de la publication
                        scientifique sous ses différents prismes et initiation à la production de
                        documents scientifiques avec le logiciel/langage de composition \LaTeX (un
                        semestre).
                    
                                \entry*[2017-2018]
                                
                            Poste de khôlleur dans deux lycées lyonnais (La Martinière-Duchère et
                        Juliette Récamier).
                    \end{rubric}

%%% rubric.tex ends here

%%% Local Variables:
%%% mode: latex
%%% TeX-master: "raw"
%%% End: