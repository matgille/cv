%%% rubric.tex --- Example of using CurVe.

%% Copyright (C) 2000, 2001, 2002, 2003, 2004, 2005, 2010 Didier Verna.

%% Author:        Didier Verna <didier@lrde.epita.fr>
%% Maintainer:    Didier Verna <didier@lrde.epita.fr>
%% Created:       Thu Dec 10 16:04:01 2000
%% Last Revision: Mon Dec  6 11:04:22 2010

%% This file is part of CurVe.

%% CurVe may be distributed and/or modified under the
%% conditions of the LaTeX Project Public License, either version 1.1
%% of this license or (at your option) any later version.
%% The latest version of this license is in
%% http://www.latex-project.org/lppl.txt
%% and version 1.1 or later is part of all distributions of LaTeX
%% version 1999/06/01 or later.

%% CurVe consists of the files listed in the file `README'.


%%% Commentary:

%% Contents management by FCM version 0.1.


%%% Code:


\begin{rubric}{Publications et communications}
                    \subrubric{Publications}
                    
                
                    
                    \entry*
                 \enquote{L'évolution du Regimiento
                        de los prínçipes (1345-1494), conditionnée par le
                        pouvoir politique ?} in \textit{Écritures du Pouvoir}, 2, éd. Véronique
                        Lamazou-Duplan, Ausonius éditions, 2019, Scripta
                        Medievalia, pp. 137-148. \href{https://hal.archives-ouvertes.fr/hal-02369116}{HAL.} 
                    
                    \entry*
                \textit{Modèle de reconnaissance optique de
                        caractères - Kraken - Incunables sévillans
                        1494-1500} (set de données). 05 février 2020.
                        \href{https://zenodo.org/record/3643393}{\includegraphics[scale=0.55]{img/kraken_doi.png}}
                \subrubric{Communications}
            
                    
                    \entry*
                Simon Gabay, Lucie Rondeau Du Noyer, Matthias
                        Gille Levenson, Ljudmila Petkovic et Alexandre Bartz,
                        \enquote{\textit{Quantifying the Unknown: How many manuscripts
                        of the marquise de Sévigné still exist?}},
                        DH2020, Ottawa, Canada. \href{https://hal.archives-ouvertes.fr/hal-02898929/document}{\textit{Abstract}} et \href{http://dx.doi.org/10.17613/2pwa-0f46}{diapositives}.
                    
                    \entry*
                 \enquote{\textit{Editing a 15th century political
                        treatise using the computer: a back and forth between
                        meaning and information}}, \textit{Iberian
                        Connections seminar}, Yale University, 12 novembre
                        2019. \href{https://iberian-connections.yale.edu/articles/editing-a-xvth-century-political-treatise-using-the-computer/}{Lien}.
                    
                    \entry*
                Paul Bertrand, Ariane Pinche, Matthias Gille
                        Levenson et Margot Ferrand, \enquote{\textit{COSME² - Complexities: 30 Years Of Research Of
                        Medievalists DH Concerning A Thousand Years Of Medieval
                        Sources}} (poster -- travail collectif).
                        Congrès \enquote{\textit{Digital Humanities 2019}}
                        (DH2019), du 9 au 12 juillet 2019, Utrecht, Pays-Bas.
                        \href{https://zenodo.org/record/3275597}{\includegraphics[scale=0.55]{img/zenodo3275597.png}}
                    
                    
                    \entry*
                 \enquote{Du concept au récit:
                        théorie politique et enxiemplo dans le
                        \enquote{Regimiento de los prínçipes} (III,
                        3)}. Colloque \enquote{Theorica 7: Politia}, ÉNS
                        de Lyon, Lyon, 6 juin 2019. 
                    
                    \entry*
                \enquote{La variance et ses enjeux dans l'édition
                        numérique du \enquote{Regimiento de los
                        príncipes}}. Journée d'étude Humanités
                        Numériques, Agorantic-Ciham, Avignon, 27
                        mai 2019.
                    
                    \entry*
                \enquote{Construire l'édition numérique d'une
                        tradition textuelle complexe: propositions pour le
                        \enquote{Regimiento de los prínçipes}}.
                        Atelier IRPALL \enquote{La fabrique du texte},
                        Toulouse, 22 février 2019.
                    
                    \entry*
                \enquote{L'évolution du \enquote{Regimiento de los
                        prínçipes} (1345-1494), au service du pouvoir
                        ?} Séminaire international de formation avancée:
                        \href{https://recherche.univ-pau.fr/fr/accueil/cpim.html}{Les cultures politiques dans la Péninsule ibérique et
                        au Maghreb}, Bielle, octobre 2018. \end{rubric}

%%% rubric.tex ends here

%%% Local Variables:
%%% mode: latex
%%% TeX-master: "raw"
%%% End: